\section{ME22B176}
I am going to explain the HAGEN-POISEUILLE'S LAW.
\begin{itemize}
\item Name: Pawar Devesh Pramod
\item Github id: Devesh176
\end{itemize}
\section{HAGEN-POISEUILLE'S LAW}
\subsection{Introduction}
This law was experimentally derived independently by G. Hagen and J. L. Poiseuille between 1838 and 1840. The assumptions of the equation are that the fluid is incompressible and Newtonian; the flow is >
\subsection{Equation}
\begin{equation}
    Q = \frac{\pi a^{4}}{8\mu} \frac{\Delta P}{l}
\end{equation}
where,
\begin{itemize}
\item $Q$ = volume flow rate $(m^{3}/s)$
\item $\Delta P$ = $P_{1} - P_{2}$ = pressure difference ($Pa$ or $N/m^{2}$ or $kg/m \cdot s^{2}$)
\item $P_{1}$ = fliud pressure at entrance to tube ($Pa$)
\item $P_{2}$ = fluid pressure at exit from tube ($Pa$)
\item $a$ = tube radius ($m$)
\item $\mu$ = fluid viscosity ($kg/m \cdot s$ or $Pa \cdot s$), and
\item $l$ = length over which the pressure drop is measured ($m$).
\end{itemize}
\footnote{Reference- H C Verma Volume 1, Chapter No.14 Some Mechanical Properties Of Matter, Page no.291.}
\end{document}

